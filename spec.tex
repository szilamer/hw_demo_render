Egy olyan webes felület létrehozása, amely lehetővé teszi a felhasználók számára:​

A beteg betegtörténeti eseményeinek időrendi megtekintését egy idővonalon keresztül.​

A betegségek és események közötti kapcsolatok vizualizálását egy gráf segítségével.​

Interaktív navigációt az idővonal és a gráf között, ahol az egyik komponens változása automatikusan frissíti a másikat.​

Hozzáférést a kapcsolódó dokumentumokhoz és egyéb funkciókhoz (chatbot, dokumentumfeltöltés, időpontfoglalás) a megfelelő szekciókban.​

🧱 Felépítés
Idővonal Sáv (Top)

Az oldal tetején, teljes szélességben.​

Betegtörténeti események időrendben történő megjelenítése.​

Eseményekhez kapcsolódó dokumentumok ikonjainak megjelenítése, amelyekre kattintva a dokumentum megnyitható.​

Interaktív idővonal, amely lehetővé teszi az időpontok közötti navigációt.​

Az idővonalon történő navigáció automatikusan frissíti a gráf vizualizációt a megfelelő csomópontokra fókuszálva.​

Gráf Vizualizáció (Középen Baloldalt)

Az oldal közepén, bal oldalon.​

Betegségek, események és azok közötti kapcsolatok vizualizálása gráf formájában.​

Interaktív csomópontok és élek, amelyekre kattintva részletes információ jelenik meg.​

Az idővonalon történő navigáció hatására a gráf automatikusan a releváns csomópontokra fókuszál.​

Lehetőség a gráf manuális navigációjára és nagyítására/kicsinyítésére.​

Chatbox (Középen Jobboldalt)

Az oldal közepén, jobb oldalon.​

(Későbbi fejlesztés) Chatbot integráció a felhasználói kérdések megválaszolására és navigáció segítésére.​

(Későbbi fejlesztés) Lehetőség a felhasználói visszajelzések gyűjtésére és feldolgozására.​

Információs/Feature Sáv (Alul)

Az oldal alján, teljes szélességben.​

(Későbbi fejlesztés) Dokumentumfeltöltési lehetőség a felhasználók számára.​

(Későbbi fejlesztés) Időpontfoglalási rendszer integrációja.​

(Későbbi fejlesztés) Egyéb információk és funkciók megjelenítése.​

🛠️ Technológiai Stack
Frontend:

React.js: A felhasználói felület komponensalapú fejlesztéséhez.​

vis.js: Interaktív idővonal és gráf vizualizációkhoz.​

Neovis.js: Neo4j adatbázis gráfjainak vizualizálásához a böngészőben.​

Backend:

Node.js: Szerveroldali JavaScript futtatókörnyezet.​

Express.js: Webalkalmazások és API-k fejlesztéséhez.​

Neo4j: Gráfadatbázis a betegtörténeti adatok és kapcsolatok tárolására.​

🔄 Interakciók és Szinkronizáció
Az idővonalon történő navigáció automatikusan frissíti a gráf vizualizációt, fókuszálva a kiválasztott időponthoz kapcsolódó csomópontokra.​

A gráfban történő navigáció (pl. csomópont kiválasztása) frissítheti az idővonalat, kiemelve a kapcsolódó eseményeket.​

Eseménykezelők és API-k biztosítják a komponensek közötti kommunikációt és adatcserét.​

Egy olyan webes felület létrehozása, amely lehetővé teszi a felhasználók számára:​

A beteg betegtörténeti eseményeinek időrendi megtekintését egy idővonalon keresztül.​

A betegségek és események közötti kapcsolatok vizualizálását egy gráf segítségével.​

Interaktív navigációt az idővonal és a gráf között, ahol az egyik komponens változása automatikusan frissíti a másikat.​

Hozzáférést a kapcsolódó dokumentumokhoz és egyéb funkciókhoz (chatbot, dokumentumfeltöltés, időpontfoglalás) a megfelelő szekciókban.​

🧱 Felépítés
Idővonal Sáv (Top)

Az oldal tetején, teljes szélességben.​

Betegtörténeti események időrendben történő megjelenítése.​

Eseményekhez kapcsolódó dokumentumok ikonjainak megjelenítése, amelyekre kattintva a dokumentum megnyitható.​

Interaktív idővonal, amely lehetővé teszi az időpontok közötti navigációt.​

Az idővonalon történő navigáció automatikusan frissíti a gráf vizualizációt a megfelelő csomópontokra fókuszálva.​

Gráf Vizualizáció (Középen Baloldalt)

Az oldal közepén, bal oldalon.​

Betegségek, események és azok közötti kapcsolatok vizualizálása gráf formájában.​

Interaktív csomópontok és élek, amelyekre kattintva részletes információ jelenik meg.​

Az idővonalon történő navigáció hatására a gráf automatikusan a releváns csomópontokra fókuszál.​

Lehetőség a gráf manuális navigációjára és nagyítására/kicsinyítésére.​

Chatbox (Középen Jobboldalt)

Az oldal közepén, jobb oldalon.​

(Későbbi fejlesztés) Chatbot integráció a felhasználói kérdések megválaszolására és navigáció segítésére.​

(Későbbi fejlesztés) Lehetőség a felhasználói visszajelzések gyűjtésére és feldolgozására.​

Információs/Feature Sáv (Alul)

Az oldal alján, teljes szélességben.​

(Későbbi fejlesztés) Dokumentumfeltöltési lehetőség a felhasználók számára.​

(Későbbi fejlesztés) Időpontfoglalási rendszer integrációja.​

(Későbbi fejlesztés) Egyéb információk és funkciók megjelenítése.​

🛠️ Technológiai Stack
Frontend:

React.js: A felhasználói felület komponensalapú fejlesztéséhez.​

vis.js: Interaktív idővonal és gráf vizualizációkhoz.​

Neovis.js: Neo4j adatbázis gráfjainak vizualizálásához a böngészőben.​

Backend:

Node.js: Szerveroldali JavaScript futtatókörnyezet.​

Express.js: Webalkalmazások és API-k fejlesztéséhez.​

Neo4j: Gráfadatbázis a betegtörténeti adatok és kapcsolatok tárolására.​

🔄 Interakciók és Szinkronizáció
Az idővonalon történő navigáció automatikusan frissíti a gráf vizualizációt, fókuszálva a kiválasztott időponthoz kapcsolódó csomópontokra.​

A gráfban történő navigáció (pl. csomópont kiválasztása) frissítheti az idővonalat, kiemelve a kapcsolódó eseményeket.​

Eseménykezelők és API-k biztosítják a komponensek közötti kommunikációt és adatcserét.​

